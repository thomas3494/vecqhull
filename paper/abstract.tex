\begin{abstract}
    For optimizing both runtime and energy usage, an application will have to
    leverage the SIMD, or vector-instructions of a processor.

    The energy consumption per clock cycle can be drastically reduced by
    simplifying the instruction set and keeping the number of registers small.
    However, this may decrease performance, requiring more cycles to solve
    a problem. For this reason an energy comparison between processors is not
    as simple as looking at the specs of the processor.

    In order to investigate these trade-offs in a realistic setting, we have
    vectorized the Quickhull algorithm. This is an important building block in
    computational geometry, and requires both significant computation, as well
    as non-trivial data-movement. It also features heavy register pressure.

    This is implemented in Highway, a C++ library that can generate vectorized
    code for a wide variety of hardware.

    We compare the energy efficiency on low-energy single-board computers for
    the RVV 1.0, avx2, avx512, and NEON instruction sets.
\end{abstract}
