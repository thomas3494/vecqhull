\section{Introduction}

The convex hull of a set of points in $\mathbb{R}^2$, can be intuitively 
understood as follows: Put a nail in a piece of wood at each point. Then span 
an elastic band around these nails. The resulting polygon is the convex hull of 
the points.

This is a fundamental algorithm in Computational Geometry with applications
in .... 

To be more precise, a convex combination of points $x_1, \cdots, x_n$ is a 
weighted average $\lambda_1 \cdot x_1 + \cdots \lambda_n \cdot x_n$, 
$\lambda_i \geq 0, \sum_{i = 1}^n \lambda_i = 1$. A set that is closed under
taking convex combinations is called convex. The convex hull of a set $X$ is
the set of all its convex combinations and denoted with $CH(X)$. For $X$
a finite set, the convex hull is a polytope with vertices in $X$.

In the planar case we would like to know the vertices and edges of this polygon.
An efficient data structure for this is to list the vertices in
(counter-)clockwise order, because the edges are given by adjacent vertices.

Finding this is the convex hull problem. There exist many algorithms for this
\cite{}, of which we find Quickhull to be the fastest (see 
Section~\ref{sec:perf}). This algorithm is similar to Quicksort in that
it recursively splits the points into two sets. Unlike Quicksort however,
more points than just the pivot may be discarded. Moreover, we are not
free to choose the pivot, which turns out to make parallelisation challenging.
Nonetheless, we can adept recent insights from the sorting community to engineer
a fast quickhull implementation for shared memory computers.

We make the following contributions.

\begin{itemize}
    \item We present a vectorized algorithm for extracting two disjoint 
          subsets in-place.
    \item We parallelise this algorithm in a bandwidth-friendly manner.
%    \item We improve the accuracy of the quickhull algorithm without using
%          more precise types. \tkcomment{TODO}
\end{itemize}

\iffalse

The convex hull of a set of points in $\mathbb{R}^2$, can be intuitively 
understood as follows: Put a nail in a piece of wood at each point. Then span 
an elastic band around these nails. The resulting shape is the convex hull of 
the points. This has as property that a line between any two points on the 
convex hull is contained within it.

To be more precise, a set $C \subseteq \mathbb{R}^2$ is convex, whenever
it is closed under taking weighted averages. In other words,
if $c_1, \cdots, c_n \in C$, then for any weights 
$\lambda_1, \cdots, \lambda_n$, satisfying $\lambda_i \geq 0$, 
$\lambda_1 + \cdots + \lambda_n = 1$, also
$\lambda_1 c_1 + \cdots + \lambda_n c_n$ must be in $C$.

The convex hull of a bounded set $P \subseteq \mathbb{R}^2$ is the smallest 
convex set $CH(P)$ containing $P$. This set can be described by a finite 
number of points, its vertices. There exist $c_1, \cdots, c_n$ such that
$$CH(P) = \left\{\sum_{i = 1}^n \lambda_i c_i \ | \ 
                      \lambda_i \geq 0, \sum_{i = 1}^n \lambda_i = 1\right\}$$

Figure~\ref{fig:convex_hull} shows an example of a set of $30$ points, having 
a convex hull of $8$ points.

\begin{figure}[ht]
\resizebox{\columnwidth}{!}{%
        \begin{tikzpicture}
            \node (26_32) at (5.200, 6.400) {$\bullet$};
            \node (26_9) at (5.200, 1.800) {$\bullet$};
            \node (27_49) at (5.400, 9.800) {$\bullet$};
            \node (37_18) at (7.400, 3.600) {$\bullet$};
            \node (9_18) at (1.800, 3.600) {$\bullet$};
            \node (16_14) at (3.200, 2.800) {$\bullet$};
            \node (40_13) at (8.000, 2.600) {$\bullet$};
            \node (22_17) at (4.400, 3.400) {$\bullet$};
            \node (34_2) at (6.800, 0.400) {$\bullet$};
            \node (37_4) at (7.400, 0.800) {$\bullet$};
            \node (8_13) at (1.600, 2.600) {$\bullet$};
            \node (10_1) at (2.000, 0.200) {$\bullet$};
            \node (38_19) at (7.600, 3.800) {$\bullet$};
            \node (47_46) at (9.400, 9.200) {$\bullet$};
            \node (38_25) at (7.600, 5.000) {$\bullet$};
            \node (11_16) at (2.200, 3.200) {$\bullet$};
            \node (7_38) at (1.400, 7.600) {$\bullet$};
            \node (27_37) at (5.400, 7.400) {$\bullet$};
            \node (37_14) at (7.400, 2.800) {$\bullet$};
            \node (5_48) at (1.000, 9.600) {$\bullet$};
            \node (34_21) at (6.800, 4.200) {$\bullet$};
            \node (14_24) at (2.800, 4.800) {$\bullet$};
            \node (35_36) at (7.000, 7.200) {$\bullet$};
            \node (43_21) at (8.600, 4.200) {$\bullet$};
            \node (40_30) at (8.000, 6.000) {$\bullet$};
            \node (25_48) at (5.000, 9.600) {$\bullet$};
            \node (46_37) at (9.200, 7.400) {$\bullet$};
            \node (49_34) at (9.800, 6.800) {$\bullet$};
            \node (8_49) at (1.600, 9.800) {$\bullet$};
            \node (31_48) at (6.200, 9.600) {$\bullet$};
            \node (26_42) at (5.200, 8.400) {$\bullet$};
            \node (14_35) at (2.800, 7.000) {$\bullet$};
            \node (32_41) at (6.400, 8.200) {$\bullet$};
            \node (22_21) at (4.400, 4.200) {$\bullet$};
            \node (7_29) at (1.400, 5.800) {$\bullet$};
            \node (19_41) at (3.800, 8.200) {$\bullet$};
            \node (1_33) at (0.200, 6.600) {$\bullet$};
            \node (16_38) at (3.200, 7.600) {$\bullet$};
            \node (19_11) at (3.800, 2.200) {$\bullet$};
            \node (9_12) at (1.800, 2.400) {$\bullet$};
            \node (42_36) at (8.400, 7.200) {$\bullet$};
            \node (10_38) at (2.000, 7.600) {$\bullet$};
            \node (25_12) at (5.000, 2.400) {$\bullet$};
            \node (22_33) at (4.400, 6.600) {$\bullet$};
            \node (13_5) at (2.600, 1.000) {$\bullet$};
            \node (31_39) at (6.200, 7.800) {$\bullet$};
            \node (0_47) at (0.000, 9.400) {$\bullet$};
            \node (24_32) at (4.800, 6.400) {$\bullet$};
            \node (40_47) at (8.000, 9.400) {$\bullet$};
            \node (6_47) at (1.200, 9.400) {$\bullet$};

            \graph {(0_47) ->[red] (8_49)};
            \graph {(8_49) ->[red] (27_49)};
            \graph {(27_49) ->[red] (47_46)};
            \graph {(47_46) ->[red] (49_34)};
            \graph {(49_34) ->[red] (37_4)};
            \graph {(37_4) ->[red] (34_2)};
            \graph {(34_2) ->[red] (10_1)};
            \graph {(10_1) ->[red] (1_33)};
            \graph {(1_33) ->[red] (0_47)};
        \end{tikzpicture}
    }
    \caption{Convex hull of $50$ points.}
    \label{fig:convex_hull}
\end{figure}

The convex hull problem is to find these vertices, and list them
in clockwise order. This problem is of direct practical application in many
areas, such as image processing \cite{Rosenfeld69}, pattern recognition
\cite{Duda74}, and packing problems \cite{Freeman75}.
\fi
