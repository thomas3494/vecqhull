\section{Conclusion}

On the performance side, we have seen that applying branchless techniques
from sorting can significantly speed up Quickhull. 
Our performance evaluation shows that there is much room for improvement still. 
One improvement that would mostly benefit the sequential implementation would 
be to vectorize by hand. For the multithreaded implementation it 
becomes important to reduce the number of passes over the data. This could be 
done by identifying more points on the hull in one pass, and employing 
techniques similar to super scalar samplesort \cite{ssss} or ips4o \cite{ips4o}. 
However, except the very first pass, it is not clear how to 
identify more points on the hull in one pass. Furthermore, our partition 
scheme potentially iterates over the data twice. It may be possible to write a 
branchless partition that only iterates over the data once.

On the energy side, ...

Akl-Toussaint could be implemented by a IPS4O-like idea to reduce the
number of passes. Finds left, right, top, bottom, smallest and largest
sum and difference of x and y coordinates.
