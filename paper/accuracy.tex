\section{Numerical Robustness}

An algorithm should not only be efficient, but also correct. 

$d(f(x), \widehat{f(x)}) \leq d(f(x), f(\hat{x})) + d(f(\hat{x}), \widehat{f(x)})$

\subsection{Robustness Framework}

In a similar vein to \cite{Jiang06}, we define error between two sets
$P$ and $P'$ as their Hausdorff distance divided by the maximum absolute
value over their $x$-coordinates and $y$-coordinates $M$. We do not require
the number of points to be the same as we have not defined the convex hull
as a minimum representation. \tkcomment{If you do not want this, you probably
want an $\epsilon$-strong convex hull, which can be obtained efficiently
from the approximate convex hull we compute.}

We write $\epsilon$ for the machine precision, $2^{-53}$ for double-precision, 
and $2^{-24}$ for single-precision. We ignore terms of order $O(\epsilon^2)$.

\subsection{Accurate Evaluation of Right-Hand Turn}

As discussed, we can decide whether $puq$ makes a right-hand turn by testing
$orient(p, u, q) > 0$. We have $orient(p, u, q) = -orient(u, p, q)$, so
we can equivalently compare

$$(p_x - u_x) \cdot (q_y - p_y) - (p_y - u_y) \cdot (q_x - p_x) < 0 \iff$$
$$(p_x - u_x) \cdot (q_y - p_y) - (p_y - u_y) \cdot (q_x - p_x) < 0 \iff$$
$$(p_x - u_x) \cdot (q_y - p_y) < (p_y - u_y) \cdot (q_x - p_x).$$

The number of operations are the same, but we evaluate this in a loop over $u$, 
so this formulaton has the performance benefit that $q_x - p_x$ and $q_y - p_y$ 
can be lifted out of the loop. We denote the true evaluation of this inequality
by $rt(p, u, q)$ and the computed evaluation by $\widehat{rt}(p, u, q)$.

\begin{lemma}\label{lem:right-turn}
    If $\widehat{rt}(p, u, q)$ is true, but $rt(p, u, q)$ is not, or the
    other way around, then $d(u, pq) < \frac{6\epsilon}{1 - 6\epsilon}M$.
\end{lemma}

\begin{proof}
    We write $fl(\cdots)$ for the floating point evaluation of a primitive 
    operation. The IEEE-754 standard satisfies
    $fl(x \circ y) = (1 + \delta)(x \circ y)$, $|\delta| < \epsilon$ for
    $\circ \in \{+, -, \cdot\}$. This gives us
    $$fl(fl(p_x - u_x) fl(q_y - p_y)) < 
            fl(fl(p_y - u_y) fl(q_x - p_x)) \iff$$
    $$(p_x - u_x) (q_y - p_y) < \frac{(1 + \delta_1)(1 + \delta_2)(1 + \delta_3)}
    {(1 + \delta_4)(1 + \delta_5)(1 + \delta_6)} (p_y - u_y) (q_x - p_x).$$

    By a non-trivial (\tkcomment{but well-known?}) result, there is some $\theta$
    such that
    $$\frac{(1 + \delta_1)(1 + \delta_2)(1 + \delta_3)}{(1 + \delta_4)(1 + \delta_5)(1 + \delta_6)} = 1 + \theta$$
    and $|\theta| \leq \frac{6\epsilon}{1 - 6\epsilon} := \gamma_6$.
    That simplifies the equation to
    $$(p_x - u_x) (q_y - p_y) < (1 + \theta) (p_y - u_y) (q_x - p_x) \iff$$
    $$-\theta (p_y - u_y) (q_x - p_x) < orient(p, u, q).$$

    As $rt(p, u, q)$ is false, we have $orient(p, u, q) \leq 0$, so
    $|orient(p, u, q)| \leq \gamma_6 |(p_y - u_y) (q_x - p_x)|$.
    We can now compute upper bound
    $$d(u, pq) = \frac{orient(p, u, q)}{2 \lVert p - q \rVert} \leq$$
    $$\frac{\gamma_6 |(p_y - u_y) (q_x - p_x)|}{2 \sqrt{(q_x - p_x)^2 + (q_y - p_y)^2}} \leq$$
    $$\frac{1}{2}\gamma_6 |p_y - u_y| \leq 
            \frac{1}{2} \gamma_6 (|p_y| + |u_y|)\gamma_6 M = \gamma_6 M.$$

    The case $rt(p, u, q)$ true but $\hat{rt}(p, u, q)$ false is analogous.
\end{proof}

\subsection{Accurate Evaluation of Distance Test}

We can test whether $u$ is farther from $pq$ than $u'$ by comparing
$orient(u, p, q)$ to $orient(u', p, q)$. So it
is tempting to compute the orientation once for each point, and then do both
the right-turn test and the distance test with this quantity, but this
runs into precision problems. We are not very much concerned with the
subtraction of coordinates because they are input values and hence not
subject to errors (at least errors we control). However, the subtraction 
in the middle does have round-off errors we introduce by the multiplications.

So instead of working with orientations directly, we use that

$$orient(p, u, q) > orient(p, u', q) \iff orient(u, p, q) < orient(u', p, q) 
\iff$$

$$(q_y - p_y) (u_x - u'_x) < (q_x - p_x) (u_y - u'_y).$$

We write $frt(p, q, u, u')$ for the predicate $d(pq, u) > d(pq, u')$,
and $\widehat{frt}(p, q, u, u')$ for the computed predicate.
\tkcomment{I guess this is not really distance, but more like signed-distance
as it can be negative if $u$ is to the other side of $pq$.}

\begin{lemma}\label{lem:farther}
    The computed predicate $\widehat{frt}$ is backward-stable in the 
    following sense. Let $M$ be an upper bound on the magnitude of 
    $x$- and $y$-coordinates of points in $P$. 
    Then if $\widehat{frt}(p, q, u, u')$ is true, there exists 
    $\hat{u}'$ such that

    $$\lVert u' - \hat{u}' \rVert < \frac{6\epsilon}{1 - 6\epsilon}M$$

    and $frt(p, q, \hat{u}, \hat{u}')$ is true. Furthermore, this 
    $\hat{u}'$ can be obtained by moving $u$ perpendicular to $pq$
    by a distance $\frac{6\epsilon}{1 - 6\epsilon}M$ or more.
\end{lemma}

\begin{proof}
    Suppose $\widehat{frt}(p, q, u, u')$ is true. Then analogously to 
    Lemma~\ref{lem:right-turn}, we have

    $$(q_y - p_y) (u_x - u'_x) - (q_x - p_x) (u_y - u'_y) < 
            \theta(q_x - p_x) (u_y - u'_y),$$

    for $|\theta| < \gamma_6$. If $frt(p, q, u, u')$ is false, then

    $$(q_y - p_y) (u_x - u'_x) - (q_x - p_x) (u_y - u'_y) \geq 0,$$

    so 

    $$|(q_y - p_y) (u_x - u'_x) - (q_x - p_x) (u_y - u'_y)| \leq 
        |\theta(q_x - p_x) (u_y - u'_y)|.$$

    We also have 

    $$(q_y - p_y) (u_x - u'_x) - (q_x - p_x) (u_y - u'_y) =
       orient(u, p, q) - orient(u', p, q) =$$
    $$2 d(u, pq) \lVert p - q \rVert - 2 d(u', pq) \lVert p - q \rVert = 
        2 \lVert p - q \rVert (d(u, pq) - d(u', pq)).$$

    Taking this together yields

    $$d(u, pq) - d(u', pq) \leq 
       |\theta|\frac{|(q_x - p_x) (u_y - u'_y)|}{2 \lVert p - q \rVert} \leq$$
    $$\frac{1}{2}|\theta| |u_y - u'_y| \leq |\theta|M.$$

    So we obtain $\hat{u}'$ by moving $u'$ perpendicular to $pq$ by at least
    $|\theta|M$.
\end{proof}

\subsection{Numerical Stability of Quickhull}

Like Quicksort, it is possible to construct an adverserial input where we have 
$n$ levels of recursion, yielding $O(n^2)$ runtime. Likewise, the forward
and backward error are also linear in the depth of the recursion.
The depth is usually $O(1)$ or $O(\log n)$, but can be $O(n)$ in the worst 
case (where the runtime also becomes unfeasible).

\begin{theorem}
    We have implemented Quickhull in a way that is numerically stable.
    The backward and forward error is bounded by $2\gamma_6d$,
    where $d$ is the depth of the recursion. 
\end{theorem}

\begin{proof}
    $\widehat{CH(S_1)} \approx CHq}$
\end{proof}

\begin{proof}
    We can view Algorithm~\ref{alg:quickhull_basic} as a single recursive 
    function that takes $P$ and three points on $CH(P)$. We initialise the
    algorithm by finding left-most and right-most points $p$, $q$ which
    are guaranteed to be on $CH(P)$. We have never assumed that the triangle
    formed by these three points is not degenerate, so we can pick `three'
    points $p, q, p$ to start. The recursive structure of the resulting 
    algorithm gives rise to an inductive argument. The induction hypothesis
    says that if there exist $\hat{p}$, $\hat{q}$, $\hat{r}$ on
    $CH(P \cup \{p, r, q\})$ satisfying $d(p, \hat{p}) < \gamma_6 M$, 
    $d(r, \hat{r}) < \gamma_6 M$, $d(q, \hat{q}) < \gamma_6 M$, the backward
    and forward error are bounded by $\gamma_6 d$. Proving this claim is enough
    because the $p$ and $q$ we use for the first recursive call are themselves
    on $CH(P)$ (they are found by coordinate comparison, which is exact).

    For both the base case and induction step, we perturb $P$ as follows.
    By Lemma~\ref{lem:right-turn}, all points that are classified incorrectly 
    have a distance of at most $\gamma_6 M$ to either $pr$ or $rq$. If 
    $u \in P$ has been incorrectly added $S_1$, we move it 
    $\frac{d(u, pr) + \gamma_6 M}{2}$ perpendicular to $pr$. This yields a 
    point $\hat{u}$ that satisfies $rt(p, u, r)$ and 
    $d(u, \hat{u}) < \gamma_6 M$ by Lemma~\ref{lem:right-turn}. We do the same 
    for points that have been incorrectly added to $S_2$. Points that were 
    incorrectly discarded are moved to $pr$ (or equivalently $rq$). As each 
    point has been moved at most $\gamma_6 M$, this yields a perturbed set 
    $P'$ with subsets $S_1', S_2'$ such that $S_1'$, $S_2'$ are exact for $P'$, 
    and $d(P, P'), d(S_1, S_1'), d(S_2, S_2') < \gamma_6$.

    \tkcomment{TODO: case $|P| \leq 1$}
    Consider the base case $d = 1$. As there are no recursive calls, we have 
    $S_1 = S_2 = \emptyset$. The construction above gives us $P'$ for which
    $\{p, r, q\}$ is at most $\gamma_6$ from some set 
    $\{\hat{p}, \hat{r}, \hat{q}\}$ that are on $CH(P' \cup \{p, r, q\})$.

    Moving on the the induction step, Lemma~\ref{lem:farther} tells us
    that there exist $\hat{r}_1$ and $\hat{r}_2$ at most $\theta_6 M$ from
    $r_1$, $r_2$ that are on the convex hull of $S_1$, $S_2$. By assumption,
    $\hat{p}$ and $\hat{q}$ are on the convex hull of $P$, That means we
    can apply the induction hypothesis. The recursive calls compute
    $H_1 = \texttt{HULL}(S_1, p, r_1, r)$ such that some $H_1'$ is
    the convex hull of some $S_1''$ where 
    $d(H_1, H_1'), d(S_1, S_1'') < \gamma_6$, and similarly for $S_2$. 

    Because of Jiang's forward error analysis, we know that 
    $d(CH(S_1'), CH(S_1)) < \sqrt{2} \epsilon$.

    Now $\hat{p} \cup H_1' \cup \hat{q} \cup \hat{H_2'}$ is the convex hull
    of ...
\end{proof}
