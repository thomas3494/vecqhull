\section{Numerical Robustness}

\tkcomment{I use a lot of big words in this section, but I don't know 
what I am doing, so probably at least half is wrong.}

\tkcomment{Insert paragraph introducing general idea and motivation behind
stability (robustness?)}

\subsection{Robustness Framework}

In a similar vein to \cite{Jiang06}, we define error between two sets
$P$ and $P'$ as their Hausdorff distance divided by the maximum absolute
value over their $x$-coordinates and $y$-coordinates $M$. We do not require
the number of points to be the same as we have not defined the convex hull
as a minimum representation. \tkcomment{If you do not want this, you probably
want an $\epsilon$-strong convex hull, which can be obtained efficiently
from the approximate convex hull we compute.}

We write $\epsilon$ for the machine precision, $2^{-53}$ for double-precision, 
and $2^{-24}$ for single-precision.

\subsection{Accurate Evaluation of Right-Hand Turn}

As discussed, we can decide whether $puq$ makes a right-hand turn by testing
$orient(p, u, q) > 0$. We have $orient(p, u, q) = -orient(u, p, q)$, so
we can equivalently compare

$$(p_x - u_x) \cdot (q_y - p_y) - (p_y - u_y) \cdot (q_x - p_x) < 0 \iff$$
$$(p_x - u_x) \cdot (q_y - p_y) - (p_y - u_y) \cdot (q_x - p_x) < 0 \iff$$
$$(p_x - u_x) \cdot (q_y - p_y) < (p_y - u_y) \cdot (q_x - p_x).$$

The number of operations are the same, but we evaluate this in a loop over $u$, 
so this formulaton has the performance benefit that $q_x - p_x$ and $q_y - p_y$ 
can be lifted out of the loop. We denote the true evaluation of this inequality
by $rt(p, u, q)$ and the computed evaluation by $\widehat{rt}(p, u, q)$.

\begin{lemma}\label{lem:right-turn}
    The computed predicate $\widehat{rt}$ is backward-stable in the 
    following sense. If $\widehat{rt}(p, u, q)$ is true, there exists 
    $\hat{u}$ such that 
    $$\lVert u - \hat{u} \rVert < \frac{6\epsilon}{1 - 6\epsilon}M$$
    and $rt(p, \hat{u}, q)$ is true. Furthermore, this $\hat{u}$ can be
    obtained by moving $u$ perpendicular to $pq$ by a distance 
    $\frac{6\epsilon}{1 - 6\epsilon}M$ or more.
\end{lemma}

\begin{proof}
    We write $fl(\cdots)$ for the floating point evaluation of a primitive 
    operation. The IEEE-754 standard satisfies
    $fl(x \circ y) = (1 + \delta)(x \circ y)$, $|\delta| < \epsilon$ for
    $\circ \in \{+, -, \cdot\}$. This gives us

    $$fl(fl(p_x - u_x) fl(q_y - p_y)) < 
            fl(fl(p_y - u_y) fl(q_x - p_x)) \iff$$
    $$(p_x - u_x) (q_y - p_y) < \frac{(1 + \delta_1)(1 + \delta_2)(1 + \delta_3)}
    {(1 + \delta_4)(1 + \delta_5)(1 + \delta_6)} (p_y - u_y) (q_x - p_x).$$

    By a non-trivial (\tkcomment{but well-known?}) result, there is some $\theta$
    such that

    $$\frac{(1 + \delta_1)(1 + \delta_2)(1 + \delta_3)}{(1 + \delta_4)(1 + \delta_5)(1 + \delta_6)} = 1 + \theta$$

    and $|\theta| \leq \frac{6\epsilon}{1 - 6\epsilon} := \gamma_6$.

    That simplifies the equation to

    $$(p_x - u_x) (q_y - p_y) < (1 + \theta) (p_y - u_y) (q_x - p_x) \iff$$
    $$-\theta (p_y - u_y) (q_x - p_x) < orient(p, u, q).$$

    If $rt(p, u, q)$ is true, we can take $\hat{u} = u$. If it is false,
    we have $orient(p, u, q) \leq 0$, so
    $|orient(p, u, q)| \leq \gamma_6 |(p_y - u_y) (q_x - p_x)|$.

    Recall that 

    $$d(u, pq) = \frac{orient(p, u, q)}{2 \lVert p - q \rVert} \leq$$
    $$\frac{\gamma_6 |(p_y - u_y) (q_x - p_x)|}{2 \sqrt{(q_x - p_x)^2 + (q_y - p_y)^2}} \leq$$
    $$\frac{1}{2}\gamma_6 |p_y - u_y| \leq 
            \frac{1}{2} \gamma_6 (|p_y| + |u_y|)\gamma_6 M = \gamma_6 M.$$

    So we can move $u$ by $\gamma_6 M$ or more to get 
    $\hat{u}$ that is truly to the left of $pq$.
\end{proof}

\subsection{Accurate Evaluation of Distance Test}

We can test whether $u$ is farther from $pq$ than $u'$ by comparing
$orient(u, p, q)$ to $orient(u', p, q)$. So from a performance standpoint it
is tempting to compute the orientation once for each point, and then do both
the right-turn test and the distance test with this quantity, but this
runs into precision problems. We are not very much concerned with the
subtraction of coordinates because they are input values and hence not
subject to errors (at least errors we control). However, the subtraction 
in the middle does have round-off errors we introduce by the multiplications.

So instead of working with orientations directly, we use that

$$orient(p, u, q) > orient(p, u', q) \iff orient(u, p, q) < orient(u', p, q) 
\iff$$

$$(q_y - p_y) (u_x - u'_x) < (q_x - p_x) (u_y - u'_y).$$

We write $frt(p, q, u, u')$ for the predicate $d(pq, u) > d(pq, u')$,
and $\widehat{frt}(p, q, u, u')$ for the computed predicate.

\begin{lemma}\label{lem:farther}
    The computed predicate $\widehat{frt}$ is backward-stable in the 
    following sense. Let $M$ be an upper bound on the magnitude of 
    $x$- and $y$-coordinates of points in $P$. 
    Then if $\widehat{frt}(p, q, u, u')$ is true, there exists 
    $\hat{u}'$ such that

    $$\lVert u' - \hat{u}' \rVert < \frac{6\epsilon}{1 - 6\epsilon}M$$

    and $frt(p, q, \hat{u}, \hat{u}')$ is true. Furthermore, this 
    $\hat{u}'$ can be obtained by moving $u$ perpendicular to $pq$
    by a distance $\frac{6\epsilon}{1 - 6\epsilon}M$ or more.
\end{lemma}

\begin{proof}
    Suppose $\widehat{frt}(p, q, u, u')$ is true. Then analogously to 
    Lemma~\ref{lem:right-turn}, we have

    $$(q_y - p_y) (u_x - u'_x) - (q_x - p_x) (u_y - u'_y) < 
            \theta(q_x - p_x) (u_y - u'_y),$$

    for $|theta| < \gamma_6$. If $frt(p, q, u, u')$ is false, then

    $$(q_y - p_y) (u_x - u'_x) - (q_x - p_x) (u_y - u'_y) \geq 0,$$

    so 

    $$|(q_y - p_y) (u_x - u'_x) - (q_x - p_x) (u_y - u'_y)| \leq 
        |\theta(q_x - p_x) (u_y - u'_y)|. $$

    We also have 

    $$(q_y - p_y) (u_x - u'_x) - (q_x - p_x) (u_y - u'_y) =
       orient(u, p, q) - orient(u', p, q) = $$
    $$2 d(u, pq) \lVert p - q \rVert - 2 d(u', pq) \lVert p - q \rVert = 
        2 \lVert p - q \rVert (d(u, pq) - d(u', pq)). $$

    Taking this together yields

    $$d(u, pq) - d(u', pq) \leq 
       |\theta|\frac{|(q_x - p_x) (u_y - u'_y)|}{2 \lVert p - q \rVert} \leq$$
    $$\frac{1}{2}|\theta| |u_y - u'_y| \leq |\theta|M.$$

    So we obtain $\hat{u}'$ by moving $u'$ perpendicular to $pq$ by at least
    $|\theta|M$.
\end{proof}

\subsection{Numerical Stability of Quickhull}

%\begin{theorem}
%    We have implemented Quickhull in a way that is stable for three-point
%    convex hulls, with an error bound of $\frac{6\epsilon}{1 - 6\epsilon}$
%    on both the forward and backward error.
%\end{theorem}
%
%\begin{proof}
%    \tkcomment{
%    For a set of points $P$, we find left-most point $p$ right-most point $q$,
%    and another point $u$ on the convex hull. Assume without loss of generality
%    that $rt(q, u', p)$ is false for all $u' \in P$.
%\end{proof}
