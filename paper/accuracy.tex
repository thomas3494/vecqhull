\section{Numerical Robustness}

In a similar vein to \cite{Jiang06}, we define error between two sets
$P$ and $P'$ as their Hausdorff distance divided by the maximum absolute
value over their $x$-coordinates and $y$-coordinates. We do not require
the number of points to be the same as we have decided to not care about
minimum representations of the convex hull.

Writing $\epsilon$ for the machine precision, $2^{-53}$ for double precision
and $2^{-24}$ for single precision, we can implement Quickhull in such a way
that both the forward and backward error are $O(m \epsilon)$ where $m$ is
the recursion depth of Algorithm~\ref{alg:quickhull_basic} (worst case $n$,
average case $\log(n)$).

\subsection{Accurate Evaluation of Right-Hand Turn}

As discussed, we can decide whether $puq$ makes a right-hand turn by testing
$orient(p, u, q) > 0$. We have $orient(p, u, q) = -orient(u, p, q)$, so
we can equivalently compare

$$(p_x - u_x) \cdot (q_y - p_y) - (p_y - u_y) \cdot (q_x - p_x) < 0 \iff$$
$$(p_x - u_x) \cdot (q_y - p_y) - (p_y - u_y) \cdot (q_x - p_x) < 0 \iff$$
$$(p_x - u_x) \cdot (q_y - p_y) < (p_y - u_y) \cdot (q_x - p_x).$$

The number of operations are the same, but we evaluate this in a loop over $u$, 
so this formulaton has the performance benefit that $q_x - p_x$ and $q_y - p_y$ 
can be lifted out of the loop. We denote the true evaluation of this inequality
by $rt(p, u, q)$ and the computed evaluation by $\widehat{rt}(p, u, q)$.


\begin{lemma}\label{lem:right-turn}
    If $\widehat{rt}(p, u, q)$ is true while $rt(p, u, q)$ is false, there exists
    $\hat{u} = u(1 + \delta)$, $|\delta| < \frac{6\epsilon}{1 - 6\epsilon}M$ 
    such that $rt(p, \hat{u}, q)$ is true.
\end{lemma}

\begin{proof}
    We write $fl(\cdots)$ for the floating point evaluation of a primitive 
    operation. The IEEE-754 standard satisfies 
    $fl(x \circ y) = (1 + \delta)(x \circ y)$, $|\delta| < \epsilon$ for
    $\circ \in \{+, -, \cdot\}$. This gives us

    $$fl(fl(p_x - u_x) fl(q_y - p_y)) < 
            fl(fl(p_y - u_y) fl(q_x - p_x)) \iff$$
    $$(p_x - u_x) (q_y - p_y) < \frac{(1 + \delta_1)(1 + \delta_2)(1 + \delta_3)}
    {(1 + \delta_4)(1 + \delta_5)(1 + \delta_6)} (p_y - u_y) (q_x - p_x).$$

    By a non-trivial (\tkcomment{but well-known?}) result, there is some $\theta$
    such that

    $$\frac{(1 + \delta_1)(1 + \delta_2)(1 + \delta_3)}{(1 + \delta_4)(1 + \delta_5)(1 + \delta_6)} = 1 + \theta$$

    for some $|\theta| \leq \frac{6\epsilon}{1 - 6\epsilon} := \gamma_6$.

    That simplifies the equation to

    $$(p_x - u_x) (q_y - p_y) < (1 + \theta) (p_y - u_y) (q_x - p_x) \iff$$
    $$-\theta (p_y - u_y) (q_x - p_x) < orient(p, u, q).$$

    The predicate $rt(p, u, q)$ is equivalent to $orient(p, u, q) \leq 0$
    So $|orient(p, u, q)| \leq \gamma_6 |(p_y - u_y) (q_x - p_x)|$.

    Recall that 

    $$d(u, pq) = \frac{orient(p, u, q)}{2 \lVert p - q \rVert} \leq$$
    $$\frac{\gamma_6 |(p_y - u_y) (q_x - p_x)|}{2 \sqrt{(q_x - p_x)^2 + (q_y - p_y)^2}} \leq$$
    $$\frac{1}{2}\gamma_6 |p_y - u_y| \leq \gamma_6 M.$$

    So we can move $u$ by $\gamma_6 M$ to get $\hat{u}$ that is truly to the 
    left of $pq$.
\end{proof}

\subsection{Accurate Evaluation of Distance Test}

We can test whether $u$ is farther from $pq$ than $u'$ by comparing
$orient(u, p, q)$ to $orient(u', p, q)$. So from a performance standpoint it
is tempting to compute the orientation once for each point, and then do both
the right-turn test and the distance test with this quantity, but this
runs into precision problems. We are not very much concerned with the
subtraction of coordinates because they are input values and hence not
subject to errors (at least errors we control). However, the subtraction 
in the middle does have round-off errors we introduce by the multiplications.

So instead of working with orientations directly, we use that

$$orient(p, u, q) > orient(p, u', q) \iff orient(u, p, q) < orient(u', p, q) 
\iff$$

$$(q_y - p_y) (u_x - u'_x) < (q_x - p_x) (u_y - u'_y).$$

\begin{lemma}
    What do we even want here? For a pair $u, u'$ it is obvious that whenever
    our evaluation says $d(u, pq) > d(u', pq)$ there exists 
    $\hat{p} = (1 + \theta)p$, $\hat{q} = (1 + \theta)q$, $|\theta| < \gamma_6$
    such that $d(u, \hat{p}\hat{q}) > d(u', \hat{p}\hat{q})$, but maybe we want
    the perturbation in $u$? Also need $d(u, pq)$ larger than ALL other $d(u', pq)$,
    and it is not obvious that these $\theta$ are the same for different $u'$.
\end{lemma}

\begin{proof}
\end{proof}
