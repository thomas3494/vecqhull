\section{Accuracy}




%Both PBBS and VQuickhull find about $11$ million points on the convex hull
%of the circle. As a circle is convex, it should find 100 million points.
%
%The Circle from PBBS is generated by taking a random vector $\vec{v}$
%in the plane, and then dividing it by its length. We have verified with
%a arbitrary precision computing that this set is not convex. 
%The reason for this, is that computing the norm of a point $(x, y)$ is 
%ill-conditioned for small $x$ or small $y$. To see this, consider perturbed 
%$\tilde{x} = x(1 + \delta_x)$ and perturbed $\tilde{y} = y(1 + \delta_y)$. 
%Then dividing $\tilde{x}$ by the norm of $(\tilde{x}, \tilde{y})$ gives
%
%$$\frac{x(1 + \delta_x)}{\sqrt{(x(1 + \delta_x))^2 + (y(1 + \delta_y))^2}} =$$
%
%$$\frac{x(1 + \delta_x)}{\sqrt{x^2 + y^2 + 2x\delta_x + 2y \delta_y + \delta_x^2 + \delta_y^2}}.$$
%
%The relative error compared to the true result $\frac{x}{\sqrt{x^2 + y^2}}$ is
%
%$$\left|\frac{\sqrt{\frac{x^2 + y^2 + 2x\delta_x + 2y \delta_y + \delta_x^2 + \delta_y^2}{x^2 + y^2}}}{1 + \delta_x}\right| = $$
%
%$$\left|\frac{\sqrt{1 + \frac{2x\delta_x + 2y \delta_y + \delta_x^2 + \delta_y^2}{x^2 + y^2}}}{1 + \delta_x}\right|.$$
%
%This is large for small $x, y$.
%
%For this reason, we offer the following alternative method for computing random
%points on the circle. We generate random 64-bit integers using Xoshiro256**,
%and convert this into a double between $0$ and $1$ by discarding the 
%least-significant $11$ bits. This is then multiplied by $2\pi$ and fed into
%$\cos$, $\sin$. The functions $\sin$ and $\cos$ are believed to compute results
%within $1$ ulp.
%
%Using this dataset, we find about $68$ million points, which is more, but 
%still not all of them.
%
%\section{Orient}
%
%We do not actually need to know the orientation. We only need to be
%able to accurately decide whether $orient(p, u, q) > orient(p, u', q)$
%and $orient(p, u, q) > 0$.
%
%Note that 
%
%$$orient(p, u, q) = -\alpha u_x + \beta u_y - \gamma$$
%
%$\alpha = p_y - q_y$, $\beta = p_x - q_x$, $\beta q_y - \alpha q_x$.
%
%Instead of comparing with zero, we can compare with gamma. We can compute
%$-\alpha u_x + \beta u_y$ within $1.5$ ULP if necessary.
%
%Also 
%
%$$orient(p, u, q) > orient(p, u', q) \iff $$
%
%$$ - \alpha u_x + \beta u_y > - \alpha u'_x + \beta u'_y \iff $$
%
%$$\beta (u_y - u'_y) > \alpha (u_x - u'_x).$$
%
%If $u$ is close to $u'$, then so is $orient(p, u, q)$ to $orient(p, u', q)$
%(after all, orient is continuous). So we could very well be unlucky and have
%this round to the same number. 
%
%But with this other way, we can really capture the difference between $u$ and
%$u'$, so I expect something better here.
%
%If $\alpha > 0$,
%
%$$ - u_x + \frac{\beta}{\alpha} u_y > \frac{\gamma}{\alpha} $$
%
%If $\alpha < 0$,
%
%$$ u_x + \frac{\beta}{-\alpha} u_y > \frac{\gamma}{-\alpha} $$
%
