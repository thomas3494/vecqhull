\section{Preliminaries}

\subsection{Quickhull}

The Quickhull algorithm makes use of two important facts 
illustrated by Figure~\ref{fig:quickhull}. First, if $p$, $q$ are in the 
convex hull, then any point $u$ with maximum distance to the line $\vec{pq}$, 
is in the convex hull as well. Secondly, any point within the triangle 
$puq$ is not in the convex hull. Only the points to the left of $pu$ and 
to the right of $uq$ still need to be considered.

\begin{figure}[ht]
    \begin{minipage}[t]{0.49\textwidth}
    % Generated with 20 points, width 5, height 8
    \resizebox{\columnwidth}{!}{%
        \begin{tikzpicture}
            \node (6_2) at (1.500, 0.800) {$\bullet$};
            \node (16_9) at (4.000, 3.600) {$\bullet$};
            \node (7_9) at (1.750, 3.600) {$\bullet$};
            \node (7_18) at (1.750, 7.200) {$\bullet$};
            \node (9_8) at (2.250, 3.200) {$\bullet$};
            \node (16_14) at (4.000, 5.600) {$\bullet$};
            \node (p) at (0.000, 5.200) {$p$};
            \node (2_7) at (0.500, 2.800) {$\bullet$};
            \node (r2) at (1.000, 0.800) {$r_2$};
            \node (7_14) at (1.750, 5.600) {$\bullet$};
            \node (8_13) at (2.000, 5.200) {$\bullet$};
            \node (0_11) at (0.000, 4.400) {$\bullet$};
            \node (r1) at (4.500, 7.600) {$r_1$};
            \node (7_6) at (1.750, 2.400) {$\bullet$};
            \node (q) at (4.500, 2.000) {$q$};
            \node (1_16) at (0.250, 6.400) {$\bullet$};
            \node (7_18) at (1.750, 7.200) {$\bullet$};
            \node (17_7) at (4.250, 2.800) {$\bullet$};
            \node (7_4) at (1.750, 1.600) {$\bullet$};
            \node (5_8) at (1.250, 3.200) {$\bullet$};

            \graph {(0_11) ->[red] (p)};
            \graph {(p) ->[red] (1_16)};
            \graph {(1_16) ->[red] (7_18)};
            \graph {(7_18) ->[red] (r1)};
            \graph {(r1) ->[red] (q)};
            \graph {(q) ->[red] (6_2)};
            \graph {(6_2) ->[red] (r2)};
            \graph {(r2) ->[red] (0_11)};
            \graph {(p) --[black] (q)};
            \graph {(p) --[black] (r1)};
            \graph {(r1) --[black] (q)};
            \graph {(q) --[black] (r2)};
            \graph {(r2) --[black] (p)};
        \end{tikzpicture}
    }
    \caption{The points $p$ and $q$ are the left-most and 
             right-most points, and therefore in the hull. The points $r_1$,
             $r_2$ are furthest from the line $pq$, and also in the hull. Any
             point within the two triangles cannot be in the convex hull.}
    \label{fig:quickhull}
    \end{minipage}
    \hfill
    \begin{minipage}{0.49\textwidth}
    \begin{algorithm}[H] % Must disable floating for minipage
    \begin{algorithmic}[1]
        \Require $P: $ set of points in $\mathbb{R}^2$ 
            and $p, r, q \in CH(P)$ in clockwise order.
        \Ensure $H: $ vertices of convex hull of $P$ in clockwise order.
        \If{$|P| \leq 1$}
            \State Let $H = P$.
        \Else
            \State $S_1 := \{u \in P \ | \ clockwise(p, u, r)\}$.
            \State $r_1 := \argmax\limits_{u \in S_1}distance(u, pr)$.
            \State $S_2 := \{u \in P \ | \ right\_turn(r, u, q)\}$
            \State $r_2 := \argmax\limits_{u \in S_2}distance(u, rq)$.
            \State $H := \{p\} \cup $ \Call{Hull}{$S_1$, $p$, $r_1$, $r$}
                   $\cup \{q\} \cup$ \Call{Hull}{$S_2$, $r$, $r_2$, $q$}
        \EndIf
    \end{algorithmic}
    \caption{Quickhull algorithm. Typically initialized with $p, q$ the 
             left-most point, $r$ the right-most.}
    \label{alg:quickhull_basic}
    \end{algorithm}
    \end{minipage}
    \caption{The Quickhull algorithm. \tkcomment{TODO: put the algorithm
             in a function without messing up the spacing.}}
\end{figure}

This idea can be applied recursively to find the complex hull, starting with
the left-most (highest in case of ties), and right-most (lowest in case of ties)
points \cite{Barber96}, which gives Algorithm~\ref{alg:quickhull_basic}.
Note that if multiple points maximize the distance to $pr$ or $rq$, we may add
a point that is on the convex hull, but not a vertex of the convex hull.
So this is not necessarily a minimal representation.

While it is intuitively clear what points are to the left of a line $pq$, or
inside a triangle $prq$, it is not straightforward to define this precisely.
For this we need a quantity called the orientation of three points.

Given three points $p, u, q$ in the plane, we have vectors
\[
    \vec{pu} = \begin{pmatrix}
        u_x - p_x \\
        u_y - p_y
    \end{pmatrix},
    \vec{uq} = \begin{pmatrix}
        q_x - u_x \\
        q_y - u_y
    \end{pmatrix}
\]

representing oriented line segments between the points, as depicted in
Figure~\ref{fig:orient1}.

\begin{figure}[ht]
    \begin{subfigure}{0.45\textwidth}
    \begin{tikzpicture}
        \node (p) at (0, 0) {$p$};
        \node (u) at (1, 2) {$u$};
        \node (q) at (2, 1) {$q$};

        \graph {(p) ->["$\vec{pu}$"] (u)};
        \graph {(u) ->["$\vec{uq}$"] (q)};
    \end{tikzpicture}
    \caption{Two vectors induced by three points in the plane. The path 
             $p, u, q$ makes a right-hand turn.}
    \label{fig:orient1}
    \end{subfigure}\hfill
    \begin{subfigure}{0.45\textwidth}
    \begin{tikzpicture}
        \node (p) at (0, 0)  {$p$};
        \node (u) at (1, 2)  {$u$};
        \node (q) at (2, 1)  {$q$};
        \node (c) at (1, -1) {};

        \graph {(p) ->["$\vec{pu}$"] (u)};
        \graph {(u) ->["$\vec{uq}$"] (q)};
        \graph {(p) --               (c)};
        \graph {(c) --               (q)};
    \end{tikzpicture}
    \caption{Parallelogram spanned by $\vec{pu}$ and $\vec{uq}$. 
             The area of this parallelogram is $|orient(p, u, q)|$.}
    \label{fig:orient2}
    \end{subfigure}
    \caption{A visualisation of the orientation of three points.}
\end{figure}

We can view these as vectors in 3D by adding a $z$-coordinate $0$. Now the
cross product $\vec{pu} \times \vec{uq}$ is equal to 

\[
    \begin{pmatrix}
        0 \\
        0 \\
        (u_x - p_x) \cdot (q_y - u_y) - (u_y - p_y) \cdot (q_x - u_x)
    \end{pmatrix}.
\]

We write $orient(p, q, u)$ for the $z$-coordinate
$(q_x - p_x) \cdot (u_y - q_y) - (q_y - p_y) \cdot (u_x - q_x)$. 
If the cross-product points up, so if $orient(p, u, q) > 0$, we make a left-hand 
turn going from $p$ to $u$ to $q$, and if this is negative, we make a right-hand
turn. If it is zero, the points $p, u, q$ are collinear. 

We colloquially say $u$ is to the left of $pq$ if $orient(p, u, q) > 0$.

The authors of \cite{quickerthanqhull} point out that we can use the 
orientation to find the point furthest from the line segment $pq$. 
The norm of the outer product
$|\vec{pq} \times \vec{qu}| = |orient(p, q, u)|$ is the area of the
parallelogram spanned by $\vec{pq}$ and $\vec{qu}$. This is also equal to
the length of $pq$ multiplied by the distance from $u$ to the line through 
$pq$. So $|orient(p, u, q)| > |orient(p, u', q)|$ if and only if $u$ is further
from $pq$ than $u'$.

This cumulates into Algorithm~\ref{alg:quickhull_basic}. 

\subsection{Perfomance Considerations}

We review some factors that are relevant to performance on a Central Processing
Units (\textit{CPU}).

\subsubsection{Branching}

Whenever a CPU encounters a conditional jump \textemdash{:~generated by 
if statements and loops in high level languages~\textemdash{} it will guess 
whether to jump or not and start executing. This is also called branching.

This mechanism makes branching practically free if the correct guesses are made.
If not, the CPU must make an expensive recovery. 

\subsubsection{Vectorisation}

A \textit{vector} or \textit{SIMD} instruction is one that operates on
multiple primitive types (such as integers or floating point numbers) at
the same time. The number of elements is called the \textit{vector width},
and is typically between $2$ and $8$ elements.

Such instructions can be used to do multiple arithmetic operations such as
\texttt{a[0] = b[0] + c[0]; a[1] = b[1] + c[1];} in one instruction, but
also data-movement such as 
\texttt{a[0] = flag[0] ? b[0] : c[0]; a[1] = flag[1] ? b[1] : c[1];}.
This reduces the number of instructions necessary, and can be used to
eliminate branches.

\subsubsection{Memory Hierarchy and Multithreading}

The speed at which we can do a computation depends on how quickly we can
execute instructions, and on how quickly we can get the necessary data loaded
in registers. We call applications limited by the former \textit{compute-bound},
and applications limited by the latter \textit{memory-bound}.

Random Access Memory (\textit{RAM}) is a separate from the CPU. The bus
connecting the CPU to RAM is shared by all cores. So while using more cores
linearly increases the computational power available, it does not increase
the available bandwidth. This can turn problems that are compute-bound on
one core memory-bound when using all available cores.

Note that this only holds for a single physical CPU. It is possible to setup
systems with multiple physical CPUs that look like a single CPU from the
programmer's perspective. In that case, using cores from the different CPUs
does increase the available bandwidth.
