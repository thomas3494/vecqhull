\section{Related and Future Work}\label{sec:related}

Finding planar convex hulls is a well-studied problem, there exist at least ten
algorithms for solving it \cite{Graham72, Jarvis73, Eddy77, Preparata77, 
Bykat78, Akl78, Andrew79, Clarkson93, Barber96, Chan96}. Putting our
work in context also suggests several opportunities for future work.

\subsection{CPU Performance}

The original authors of Quickhull, the Computational Geometry Algorithms
Library (CGAL) \cite{CGAL}, and the Problem Based Benchmark Suite (PBBS) 
\cite{pbbs} provide implementations of Quickhull. CGAL also provides 
implementations for some of the other algorithms.
We did some preliminary testing on a laptop (Table~\ref{table:reference}).
Based on these results we have decided to choose the Quickhull algorithm and
compare against the implementation of PBBS.

\begin{table}[ht]
    \caption{Runtime in seconds for a disk of $10^7$ points}
    \label{table:reference}
    \begin{tabular}{c | c }
     Implementation & Runtime \\ 
     \hline \\
     PBBS & 0.31 \\  
     CGAL Quickhull & 0.61 \\
     CGAL Akl-Toussaint & 0.60 \\
     CGAL Bykat & 0.73 \\
     CGAL Eddy & 0.98 \\
     CGAL Graham & 1.3 \\
     CGAL Jarvis & 208 \\
     Qhull & 1.1 \\
    \end{tabular}
\end{table}

As mentioned in Section~\ref{sec:evaluation}, the performance of vqhull cannot
be improved much without reducing the needed bandwidth. This can be done
with the Akl-Toussaint heuristic. As Quickhull resembles Quicksort, this 
heuristic resembles Samplesort. It may be possible to adept the branchless 
implementation of ips4o \cite{ips4o}. The work on Quicksort from \cite{Bramas17}
and \cite{francis92} were major inspirations for this algorithm.

\subsection{Precision}

Though we have verified our output with the output of competing implementations,
finding only $10\%$ of a shape that should be convex is surprising. The field
of numerical analysis 
\textemdash{}~see \cite{Higham02} for a thorough treatise~\textemdash{} gives
us a rigourous way of reasoning about imprecision. It is important to note that
this is not only about roundoff errors in floating point arithmetic. Input data
in engineering is often imprecise due to physical measurements. 
For any problem, imprecision in the input affects the output, regardless of what
algorithm is used. The extent of this can be quantified by use of a
\textit{condition number}, which tells us what the maximum attainable precision
for our problem.
Though this is usually applied to linear algebra, the authors of
\cite{Jiang06} show that this analysis can also be applied to convex hulls,
and that we should be able to expect accurate results (the condition number
is $\sqrt{2}$). Not all algorithms achieve this accuracy \cite{Kettner08},
but a modification of Graham Scan \cite{Fortune89} can.
As far as the authors are aware, there is no analysis for Quickhull,
which is an opportunity for future work.

\subsection{GPU Performance}

There is some work on computing convex hulls on Graphic Processor Units (GPUs)
\cite{Srungarapu11}. They report a $16\times$ speedup compared to Qhull, which
is not competetive with vqhull. However, single-threaded performance has not
increased nearly at the rate of GPU performance, so this may have changed. Their
implementation is not publically available.

Quicksort is not the most suitable sorting algorithm for GPUs. We would like
to suggest the following approach for a GPU implementation. Graham Scan consists
of a $O(n \log n)$ sorting step, followed by a sequential step. This means that
the majority of work can be done by one of the readily available and tuned
sorting libraries. A preprocessing step such as Akl-Toussaint can speedup 
distributions that contain many interior points. Parallelising the sequential
step has been studied in the context of CRCW machines \cite{Goodrich87},
which are similar to GPUs in their massive parallelism.
