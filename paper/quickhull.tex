\section{Quickhull}

\tkcomment{TODO: make the notation consistent}

The Quickhull algorithm makes use of two important facts illustrated by 
Figure~\ref{fig:quickhull}. Firstly, if $p$, $q$ are in the convex hull, then 
any point $u$ with maximum distance to the line $\vec{pq}$, is in the convex 
hull as well. Secondly, any point within the triangle $pqu$ is not in the 
convex hull. Only the points to the left of $pu$ and to the right of $uq$ 
still need to be considered.

\begin{figure}[ht]
\resizebox{\columnwidth}{!}{%
        \begin{tikzpicture}
            \node (26_32) at (5.200, 6.400) {$\bullet$};
            \node (26_9) at (5.200, 1.800) {$\bullet$};
            \node (27_49) at (5.400, 9.800) {$\bullet$};
            \node (37_18) at (7.400, 3.600) {$\bullet$};
            \node (9_18) at (1.800, 3.600) {$\bullet$};
            \node (16_14) at (3.200, 2.800) {$\bullet$};
            \node (40_13) at (8.000, 2.600) {$\bullet$};
            \node (22_17) at (4.400, 3.400) {$\bullet$};
            \node (34_2) at (6.800, 0.400) {$\bullet$};
            \node (37_4) at (7.400, 0.800) {$\bullet$};
            \node (8_13) at (1.600, 2.600) {$\bullet$};
            \node (r1) at (2.000, 0.200) {$r_1$};
            \node (38_19) at (7.600, 3.800) {$\bullet$};
            \node (r2) at (9.400, 9.200) {$r_2$};
            \node (38_25) at (7.600, 5.000) {$\bullet$};
            \node (11_16) at (2.200, 3.200) {$\bullet$};
            \node (7_38) at (1.400, 7.600) {$\bullet$};
            \node (27_37) at (5.400, 7.400) {$\bullet$};
            \node (37_14) at (7.400, 2.800) {$\bullet$};
            \node (5_48) at (1.000, 9.600) {$\bullet$};
            \node (34_21) at (6.800, 4.200) {$\bullet$};
            \node (14_24) at (2.800, 4.800) {$\bullet$};
            \node (35_36) at (7.000, 7.200) {$\bullet$};
            \node (43_21) at (8.600, 4.200) {$\bullet$};
            \node (40_30) at (8.000, 6.000) {$\bullet$};
            \node (25_48) at (5.000, 9.600) {$\bullet$};
            \node (46_37) at (9.200, 7.400) {$\bullet$};
            \node (q) at (9.800, 6.800) {$q$};
            \node (8_49) at (1.600, 9.800) {$\bullet$};
            \node (31_48) at (6.200, 9.600) {$\bullet$};
            \node (26_42) at (5.200, 8.400) {$\bullet$};
            \node (14_35) at (2.800, 7.000) {$\bullet$};
            \node (32_41) at (6.400, 8.200) {$\bullet$};
            \node (22_21) at (4.400, 4.200) {$\bullet$};
            \node (7_29) at (1.400, 5.800) {$\bullet$};
            \node (19_41) at (3.800, 8.200) {$\bullet$};
            \node (1_33) at (0.200, 6.600) {$\bullet$};
            \node (16_38) at (3.200, 7.600) {$\bullet$};
            \node (19_11) at (3.800, 2.200) {$\bullet$};
            \node (9_12) at (1.800, 2.400) {$\bullet$};
            \node (42_36) at (8.400, 7.200) {$\bullet$};
            \node (10_38) at (2.000, 7.600) {$\bullet$};
            \node (25_12) at (5.000, 2.400) {$\bullet$};
            \node (22_33) at (4.400, 6.600) {$\bullet$};
            \node (13_5) at (2.600, 1.000) {$\bullet$};
            \node (31_39) at (6.200, 7.800) {$\bullet$};
            \node (p) at (0.000, 9.400) {$p$};
            \node (24_32) at (4.800, 6.400) {$\bullet$};
            \node (40_47) at (8.000, 9.400) {$\bullet$};
            \node (6_47) at (1.200, 9.400) {$\bullet$};

            \graph {(p) --[red] (q)};
            \graph {(p) --[red] (r1)};
            \graph {(r1) --[red] (q)};
            \graph {(p) --[red] (r2)};
            \graph {(r2) --[red] (q)};
        \end{tikzpicture}
    }
    \caption{The points $p$ and $q$ are the left-most and 
             right-most points, and therefore in the hull. The points $r_1$,
             $r_2$ are furthest from the line $pq$, and also in the hull. Any
             point within the two triangles cannot be in the convex hull.}
    \label{fig:quickhull}
\end{figure}

This idea can be applied recursively to find the complex hull, starting with
the left-most (highest in case of ties), and right-most (lowest in case of ties)
points.

\subsection{Orientation}

While it is intuitively clear what points are to the left of a line $pq$, or
inside a triangle $prq$, it is not straightforward to define this precisely.
For this we need a quantity called the orientation of three points.

Given three points $p, u, q$ in the plane, we have vectors
\[
    \vec{pq} = \begin{pmatrix}
        q_x - p_x \\
        q_y - p_y
    \end{pmatrix},
    \vec{qu} = \begin{pmatrix}
        u_x - q_x \\
        u_y - q_y
    \end{pmatrix}
\]

representing oriented line segments between the points, as depicted in
Figure~\ref{fig:orient1}.

\begin{figure}[ht]
    \begin{tikzpicture}
        \node (p) at (0, 0) {$p$};
        \node (q) at (1, 2) {$q$};
        \node (u) at (2, 1) {$u$};

        \graph {(p) ->["$\vec{pq}$"] (q)};
        \graph {(q) ->["$\vec{qu}$"] (u)};
    \end{tikzpicture}
    \caption{Two vectors induced by three points in the plane. The path 
             $p, q, u$ makes a right-hand turn.}
    \label{fig:orient1}
\end{figure}

We can view these as vectors in 3D by adding a $z$-coordinate $0$. Now the
cross product $\vec{pq} \times \vec{qu}$ is equal to 

\[
    \begin{pmatrix}
        0 \\
        0 \\
        (q_x - p_x) \cdot (u_y - q_y) - (q_y - p_y) \cdot (u_x - q_x)
    \end{pmatrix}.
\]

We write $orient(p, q, u)$ for the $z$-coordinate
$(q_x - p_x) \cdot (u_y - q_y) - (q_y - p_y) \cdot (u_x - q_x)$. 
If the cross-product points up, so if $orient(p, q, u) > 0$, we make a left-hand 
turn going from $p$ to $q$ to $u$, and if this is negative, we make a right-hand
turn. If it is zero, the points $p, q, u$ are colinear. 

We colloquially say $u$ is to the left of $pq$ if $orient(p, u, q) > 0$.

The authors of \cite{quickerthanqhull} point out that we can use the 
orientation to find the point furthest from the line segment $pq$. 
The norm of the outerproduct
$|\vec{pq} \times \vec{qu}| = |orient(p, q, u)|$ is the area of the
parallelogram spanned by $\vec{pq}$ and $\vec{qu}$. This is also equal to
the length of $pq$ multiplied by the distance from $u$ to the line through 
$pq$. So $|orient(p, q, u)| > |orient(p, q, u')|$ if and only if $u$ is further
from $pq$ than $u'$. \tkcomment{For greater accuracy we probably want to compare
orient(p, q, u - u'). orient is well-conditioned, but |orient(p, u, q)| could
be far larger than |orient(p, u - u', q)|. So even if the relative error of
orient(p, u, q) is small, it could lead to a misclassification.}

\begin{figure}[ht]
    \begin{tikzpicture}
        \node (p) at (0, 0)  {$p$};
        \node (q) at (1, 2)  {$q$};
        \node (u) at (2, 1)  {$u$};
        \node (c) at (1, -1) {};

        \graph {(p) ->["$\vec{pq}$"] (q)};
        \graph {(q) ->["$\vec{qu}$"] (u)};
        \graph {(p) --               (c)};
        \graph {(c) --               (u)};
    \end{tikzpicture}
    \caption{Parallellogram spanned by $\vec{pq}$ and $\vec{qu}$. 
             The area of this parallelogram is $|orient(p, q, u)|$.}
    \label{fig:orient2}
\end{figure}

This cumulates into Algorithm~\ref{alg:quickhull_basic}. By slight abuse
of notation, we assume that the sets are ordered, so that in $\{p\} \cup \{q\}$,
the point $q$ comes after $p$.

\begin{algorithm}[ht]
\begin{algorithmic}[1]
    \caption{Quickhull algorithm}\label{alg:quickhull_basic}
    \Require $P: $ set of points in $\mathbb{R}^2$,
    \Ensure $H: $ vertices of convex hull of $P$ in clockwise order.
    \State Let $p$ be the left-most point in $P$, $q$ the right-most point.
    \State Let $S_1 = \{u \in P \ | \ orient(p, u, q) > 0\}$
    \State Let $r_1 = \argmax\limits_{u \in S_1}orient(p, u, q)$.
    \State Let $S_2 = \{u \in P \ | \ orient(q, u, p) > 0\}$
    \State Let $r_2 = \argmax\limits_{u \in S_2}orient(q, u, p)$.
    \State Let $H = \{p\} \cup $ \Call{FindHull}{$S_1$, $p$, $r_1$, $q$}
            $\cup \{q\} \cup$ \Call{FindHull}{$S_2$, $q$, $r_2$, $p$}.
    \Function{FindHull}{P, p, r, q}
        \If{$|P| \leq 1$}
            \State Let $H = P$.
        \Else
            \State Let $S_1 = \{u \in P \ | \ orient(p, u, r) > 0\}$.
            \State Let $r_1 = \argmax\limits_{u \in S_1}orient(p, u, r)$.
            \State Let $S_2 = \{u \in P \ | \ orient(r, u, q) > 0\}$
            \State Let $r_2 = \argmax\limits_{u \in S_2}orient(r, u, q)$.
            \State Let $H = \{p\} \cup $ \Call{FindHull}{$S_1$, $p$, $r_1$, $r$}
                   $\cup \{q\} \cup$ \Call{FindHull}{$S_2$, $r$, $r_2$, $q$}
        \EndIf
    \EndFunction
\end{algorithmic}
\end{algorithm}

\subsection{Accuracy}

\tkcomment{Before I forget: right now we need orient(p, p, q) = orient(p, q, q) = 0
in order for the algorithm to terminate. This forces us to evaluate the orientation
the way described. However, this leads to us finding only 10\% of the points out of
100M. That is laughable as we have 53 bits of mantissa. By finding the index of $r_i$
instead of $r_i$, we can eliminate $r_i$ before calling the recursive function,
which guarantees algorithm termination. We then have free reign into evalulating
orient accurately. There is much literature on this. Search terms: Kahan Determinant
FMA, accurate sign}

\tkcomment{I think this one should work reasonably well}

We have

$$orient(p, u, q) > orient(p, u', q) \iff $$ 

$$(p_y - q_y) \cdot (u_x - u'_x) - (p_x - q_x) \cdot (u_y - u'_y) > 0$$


